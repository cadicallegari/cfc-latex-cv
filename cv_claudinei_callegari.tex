% LaTeX file for resume
% This file uses the resume document class (res.cls)

\documentclass[margin]{res}
\usepackage [brazil]{babel}     % nomes e hifenaçã em português

\usepackage{t1enc}              % Permite digitar os acentos de forma normal
\usepackage[utf8]{inputenc}

\topmargin=-0.5in  % start text higher on the page
\setlength{\textheight}{10in} % increase text height to fit resume on 1 page
\begin{document}
\name{\textit{Claudinei Callegari}}

\address{Florianópolis, SC \\ cadicallegari@gmail.com \\ Telefone: (48) 9826 4067}


\begin{resume}

\section{Resumo}       Formado em Ciência da Computação, entusiasta de software livre, com experiência em desenvolvimento de software, tendo experiência em algumas tecnologias, tais como, Python, Ruby, Java, C e Lua. Com bom relacionamento interpessoal e facilidade em trabalhar em equipe. Estou sempre a procura de oportunidades e novos desafios.
Atuando atualmente como Full Stack Developer utilizando tecnologias como Ruby, NodeJS, AngularJS, Python e Android.

\section{Educação}	Universidade Estadual do Oeste do Paraná, Bacharel em Ciência da Computação, Dezembro 2011.

\section{Experiências}

\vspace{-0.1in}
   \begin{tabbing}
   \hspace{2.3in}\= \hspace{1.7in}\= \kill % set up two tab positions
    \textbf{DBA Tecnologia}    \>\>\textbf{Maio 2013 - Atual}\\
    \textit{Software Engineer - Full Stack}\\
    \textbf{Principais Tecnologias}: Ruby, Rails, Android, Postgresql, Grape framework, Linux,\\ Python, DBus, Amazon AWS, Dokku, REST.
   \end{tabbing}\vspace{-20pt}      % suppress blank line after tabbing
    \vspace{2mm}
    
	Envolvido em todas etapas dos projetos, desde levantamento de requisitos até o suporte ao usuário, atuando principalmente no desenvolvimento com os frameworks Rails e Grape, modelando banco de dados, escrevendo testes, implementando aplicativos Android e frontend com AngularJS.

Participei da definição e do desenvolvimento de REST APIs e microservices, utilizando Ruby/Grape, para comunicação com devices; bem como dos aplicativos Android e dos módulos Python que rodam no em outros dispositivos e consomem tais serviços.


\vspace{-0.1in}
   \begin{tabbing}
   \hspace{2.3in}\= \hspace{1.7in}\= \kill % set up two tab positions
    \textbf{Digitro}    \>\>\textbf{Set 2012 - Maio 2013}\\
    \textit{Desenvolvedor}\\
    \textbf{Principais Tecnologias}: Linux, Windows, C, SDL;
   \end{tabbing}\vspace{-20pt}      % suppress blank line after tabbing
    \vspace{2mm}
    
    Implementação de novas funcionalidades e manutenção do modulo de backup do EasyCall, solução para callcenters, utilizando a linguagem C e SDL (Specification and Description Language).

   \begin{tabbing}
   \hspace{2.3in}\= \hspace{1.7in}\= \kill % set up two tab positions
    \textbf{Nexxera}    \>\>\textbf{Jul 2012 - Set 2012}\\
    \textit{Desenvolvedor}\\
    \textbf{Principais Tecnologias}: Linux e Python;
   \end{tabbing}\vspace{-20pt}      % suppress blank line after tabbing
    \vspace{2mm}
    Desenvolvimento e manutenção de aplicações para converter `layouts` de arquivos para troca de informações entre bancos e empresas, e vise e versa, utilizando Python e ferramenta proprietária.

   %\vspace{20mm}
   \begin{tabbing}
   \hspace{2.3in}\= \hspace{1.5in}\= \kill % set up two tab positions
    \textbf{Ahgora}    \>\>\textbf{Jan 2012 - Jul 2012}\\
    \textit{Desenvolvedor}\\
    \textbf{Principais Tecnologias}: C, Lua, Linux e Java(Android);
   \end{tabbing}\vspace{-20pt}      % suppress blank line after tabbing
    \vspace{2mm}
    
       Desenvolvimento de módulos para sistemas embarcados utilizando C e Lua. Desenvolvimento de aplicações Android.

   %\vspace{20mm}
   \begin{tabbing}
   \hspace{2.3in}\= \hspace{1.5in}\= \kill % set up two tab positions
    \textbf{ITAI}    \>\>\textbf{Maio 2009 - Nov 2011}\\
    \textit{Estagiário}\\
    \textbf{Principais Tecnologias}: Linux, C/C++ e Java, Valgrind;
   \end{tabbing}\vspace{-20pt}      % suppress blank line after tabbing
    \vspace{2mm}
       Desenvolvimento de módulo de configuração e comunicação utilizando C, C++. Comunicação utilizando SUN-RPC e JSON. Também utilizado as bibliotecas Libxml2 e Glib.


\section{Mais informações}
    \begin{itemize}
        \item \textbf{Linkedin}: https://br.linkedin.com/in/cadicallegari
        \item \textbf{Github}: https://github.com/cadicallegari
    \end{itemize}


\end{resume}
\end{document}
